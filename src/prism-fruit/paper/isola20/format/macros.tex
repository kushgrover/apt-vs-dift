% == TODOs
\newcommand{\todojan}[1]{\todo{\color{blue}\textbf{Jan:\\}#1}}
\newcommand{\todojulia}[1]{\todo{\highlight{Ju:\\}#1}}
\newcommand{\todopranav}[1]{\todo{\textbf{Pranav:\\}#1}}
\newcommand{\todomaxi}[1]{\todo{\textbf{Maxi:\\}#1}}

\newcommand{\todoin}[1]{{\color{red}#1}}

% == HIGHLIGHTING

% highlight for text mode
\newcommand{\highlighttext}[1]{\colorbox{black!15}{#1}}

% highlight for math mode
\newcommand{\highlight}[1]{\colorbox{black!15}{$\displaystyle#1$}}

% == ALGORITHMS
\renewcommand{\algorithmicrequire}{\textbf{Input:}}
\renewcommand{\algorithmicensure}{\textbf{Output:}}
%\algrenewcommand{\algorithmiccomment}[1]{\hskip1.5em \textbackslash *  #1 * \textbackslash}
%\renewcommand{\algorithmiccomment}[1]{\bgroup\hfill\tiny//~#1\egroup}

%define a marking command
\newcommand*{\tikzmk}[1]{\tikz[remember picture,overlay,] \node (#1) {};\ignorespaces}
%define a boxing command, argument = colour of box
\newcommand{\boxit}[1]{\tikz[remember picture,overlay]{\node[yshift=3pt,fill=#1,opacity=.25,fit={(A)($(B)+(.95\linewidth,.8\baselineskip)$)}] {};}\ignorespaces}
%define some colours according to algorithm parts (or any other method you like)
\colorlet{pink}{red!40}
\colorlet{blue}{cyan!60}

% == TABLES
\newcolumntype{L}{X}
\newcolumntype{R}{>{\raggedleft\arraybackslash}X}
\newcolumntype{C}{>{\centering\arraybackslash}X}

% Hidden column
\newcolumntype{H}{>{\setbox0=\hbox\bgroup}c<{\egroup}@{}}

% == GRAPHICS
\DeclareGraphicsExtensions{.pdf, .png, .jpg}

% == LATEX
\newcommand\numberthis{\addtocounter{equation}{1}\tag{\theequation}}

% == THEOREMS
%\newtheorem{theo}{Theorem}[section]
\newtheorem{cor}{Corollary}%[lemma]
\newtheorem{ass}{Assumption}%[lemma]
%\newtheorem{lem}[theo]{Lemma}
  
\def\qedtriangle{\hspace{\stretch1}\ensuremath\triangleleft}
\def\qedsquare{\hspace{\stretch1}\ensuremath\square}

% == SPACE

%\usepackage{microtype}
%\renewcommand{\baselinestretch}{0.975} % >=0.97

%\newcommand{\spacefu}{\vspace*{-1.2em}}
%\newcommand{\spacefl}{\vspace*{-0.8em}}
\newcommand{\myspace}{\vspace*{-0.5em}}

% == TODO

\NewDocumentCommand{\todo}{m}{%
	% Add to todo list
	\begin{tikzpicture}[remember picture, baseline=-0.75ex]%
	\node [coordinate] (inText) {};%
	\end{tikzpicture}%
	%
	% Make the margin par
	\marginpar{%
		\begin{tikzpicture}[remember picture, font=\scriptsize]%
		\draw node[draw=red, text width = 3.5cm, inner sep=0.3mm] (inNote){#1};%
		\end{tikzpicture}%
	}%
	\begin{tikzpicture}[remember picture, overlay]%
	\draw[draw=red]%
	([yshift=-0.2cm] inText)%
	% -| ([xshift=-0.05cm] inNote.west)%
	-| (inNote.south);%
	\end{tikzpicture}%
}%
% == SYMBOLS

\RequirePackage{marvosym}
\providecommand{\contradiction}{\Lightning}


% == BOOLEAN

\newcommand{\boolor}{\vee}
\newcommand{\booland}{\wedge}
\newcommand{\boolOr}{\bigvee}
\newcommand{\boolAnd}{\bigwedge}
\newcommand{\boolnot}{\lnot}


% == BASIC OPERATORS

\DeclarePairedDelimiter{\delimabs}{\lvert}{\rvert}
\DeclarePairedDelimiter{\delimnorm}{\lVert}{\rVert}
\DeclarePairedDelimiter{\delimpospart}{\lgroup}{\rgroup^+}
\DeclarePairedDelimiter{\delimnegpart}{\lgroup}{\rgroup^-}
\DeclarePairedDelimiterX{\deliminner}[2]{\lange}{\rangle}{#1, #2}
\DeclarePairedDelimiter{\delimcardinality}{\lvert}{\rvert}
\DeclarePairedDelimiter{\delimset}{\lbrace}{\rbrace}
\DeclarePairedDelimiter{\delimtuple}{(}{)}
\DeclarePairedDelimiter{\delimlistt}{[}{]}
\DeclarePairedDelimiter{\delimfun}{(}{)}

\NewDocumentCommand{\abs}{sm}{\IfBooleanTF{#1}{\delimabs{#2}}{\delimabs*{#2}}}
\NewDocumentCommand{\norm}{sm}{\IfBooleanTF{#1}{\delimnorm{#2}}{\delimnorm*{#2}}}
\NewDocumentCommand{\pospart}{sm}{\IfBooleanTF{#1}{\delimpospart{#2}}{\delimpospart*{#2}}}
\NewDocumentCommand{\negpart}{sm}{\IfBooleanTF{#1}{\delimnetpart{#2}}{\delimnetpart*{#2}}}
\NewDocumentCommand{\inner}{sm}{\IfBooleanTF{#1}{\deliminner{#2}}{\deliminner*{#2}}}
\NewDocumentCommand{\cardinality}{sm}{\IfBooleanTF{#1}{\delimcardinality{#2}}{\delimcardinality*{#2}}}
\NewDocumentCommand{\set}{sm}{\IfBooleanTF{#1}{\delimset*{#2}}{\delimset{#2}}}
\NewDocumentCommand{\tuple}{sm}{\IfBooleanTF{#1}{\delimtuple{#2}}{\delimtuple*{#2}}}
\NewDocumentCommand{\closure}{sm}{\IfBooleanTF{#1}{\delimclosure{#2}}{\delimclosure*{#2}}}
\NewDocumentCommand{\listt}{sm}{\IfBooleanTF{#1}{\delimlistt{#2}}{\delimlistt*{#2}}}
\NewDocumentCommand{\fun}{smm}{\IfBooleanTF{#1}{{#2}\delimfun{#3}}{{#2}\delimfun*{#3}}}
\NewDocumentCommand{\funMacro}{smm}{\IfNoValueTF{#3}{#1}{\fun{#2}{#3}}}

\DeclareMathOperator{\ExistsOp}{\exists}
\DeclareMathOperator{\ForallOp}{\forall}

\NewDocumentCommand{\Exists}{gg}{\IfNoValueTF{#1}{\ExistsOp}{\ExistsOp #1. \, #2}}
\NewDocumentCommand{\Forall}{gg}{\IfNoValueTF{#1}{\ForallOp}{\ForallOp #1. \, #2}}

\DeclarePairedDelimiter\ceil{\lceil}{\rceil}
\DeclarePairedDelimiter\floor{\lfloor}{\rfloor}


% == SETS OPERATORS

\newcommand{\setcomplement}[1]{{#1}^c}
\newcommand{\powerset}[1]{\mathcal{P}(#1)}

\newcommand{\unionSym}{\cup}
\newcommand{\unionBin}{\mathbin{\unionSym}}
\newcommand{\strictunionSym}{\dot{\unionSym}}
\newcommand{\strictunionBin}{\mathbin{\dot{\unionSym}}}
\newcommand{\intersectionSym}{\cap}
\newcommand{\intersectionBin}{\mathbin{\intersectionSym}}
\newcommand{\UnionSym}{\bigcup}
\newcommand{\UnionBin}{\mathbin{\UnionSym}}
\newcommand{\strictUnionSym}{\dot{\UnionSym}}
\newcommand{\strictUnionBin}{\mathbin{\strictUnionSym}}
\newcommand{\IntersectionSym}{\bigcap}
\newcommand{\IntersectionBin}{\mathbin{\IntersectionSym}}

\newcommand{\union}{\unionBin}
\newcommand{\strictunion}{\strictunionBin}
\newcommand{\intersection}{\intersectionBin}
\newcommand{\Union}{\UnionSym}
\newcommand{\strictUnion}{\strictUnionSym}
\newcommand{\Intersection}{\IntersectionSym}


% == BASIC SETS

\newcommand{\Continuous}{C}
\newcommand{\Sobolev}{\mathcal{W}}
\newcommand{\Naturals}{\mathbb{N}}
\newcommand{\Domain}{\mathfrak{D}}
\newcommand{\Measures}{\mathcal{M}}
\newcommand{\Lebesgue}{\mathcal{L}}
\newcommand{\Hilb}{H}
\newcommand{\Reals}{\mathbb{R}}
\newcommand{\Orlicz}{{\tilde{\Lebesgue}}}
\newcommand{\Distributions}{\mathcal{D}}


% == LIMITS

\NewDocumentCommand{\convto}{G{}}{\xrightarrow{#1}}
\NewDocumentCommand{\weakto}{G{}}{\xrightharpoonup{#1}}
\NewDocumentCommand{\weakstarto}{G{}}{\xrightharpoonup[*]{#1}}

% == OPERATORS

\newcommand{\gradient}{\nabla}
\newcommand{\laplacian}{\Delta}
 \DeclareDocumentCommand{\diff}{D<>{} O{}  D(){}}{\Delta_{#1}^{#2}\ifthenelse{\isempty{#3}}{}{(#3)}}
\newcommand{\boundary}{\partial}
\DeclareMathOperator{\divergence}{div}
\DeclareMathOperator{\distance}{dist}
\DeclareMathOperator{\esssup}{ess sup}
\DeclareMathOperator{\supp}{supp}
\DeclareMathOperator{\capacity}{cap}
\DeclareMathOperator{\signum}{signum}
\DeclareMathOperator{\id}{id}
\DeclareMathOperator{\const}{const}
\DeclareMathOperator{\loc}{loc}
\DeclareMathOperator*{\argmax}{arg\, max}
\DeclareMathOperator*{\argmin}{arg\, min}
\DeclareDocumentCommand{\post}{D<>{} O{} D(){}}{\mathsf{Post}_{#1}^{#2}\ifthenelse{\isempty{#3}}{}{(#3)}}
\DeclareMathOperator{\leaves}{\mathbin{\mathop{exits}}}
\newcommand{\leaving}{exiting}
\DeclareMathOperator{\stays}{\mathbin{\mathop{\mathsf{stays\_in}}}}
\newcommand{\eqdef}{\vcentcolon=}
\newcommand{\defeq}{=\vcentcolon}

\newcommand{\qee}{\hfill$\triangle$} % quod erat exemplandum % TODO del

% == LOGIC

\newcommand{\ltl}{\operatorname{LTL}}
\newcommand{\ctl}{\operatorname{CTL}}
\newcommand{\pctl}{\operatorname{pCTL}}
\newcommand{\Next}{\varbigcirc}
\newcommand{\until}{\, \mathcal{U}}
\newcommand{\wrel}{\mathcal{W}}
\newcommand{\lang}[1]{\mathcal{L}(#1)}
\newcommand{\reach}{\Diamond}
\newcommand{\alws}{\Box}
\newcommand{\true}{\mathsf{true}}
\newcommand{\false}{\mathsf{false}}
\newcommand{\turn}{\mathsf{turn}}
\newcommand{\PQ}{\mathrm{PQ}}

% == COMPLEXITY
\newcommand{\NP}{\mathbf{NP}}
\newcommand{\NEXP}{\mathbf{NEXP}}
\newcommand{\PSPACE}{\mathbf{PSPACE}}

% == PROBABILITY
\NewDocumentCommand{\distributions}{d()}{\funMacro{\mathcal{D}}{#1}}
\newcommand{\dirac}[1]{\delta_{#1}}
\newcommand{\probability}{\mathbb{P}}
\newcommand{\expectation}{\mathbb{E}}

% == MC, MPD, Games

\newcommand{\gain}{g} %value
\newcommand{\bias}{b}
\newcommand{\stat}{q}
\newcommand{\outcomes}{\mathcal{O}}
\newcommand{\salgebra}{\mathcal{F}}
\newcommand{\mevent}{\mathcal{E}}
\newcommand{\pmin}{p_{\min}}
\newcommand{\rmax}{r_{\max}}
\newcommand{\MECs}{\mathsf{MEC}}
\newcommand{\GCs}{\mathsf{GC}}
\DeclareDocumentCommand{\val}{D<>{} O{}  D(){} t'}{\mathsf{V}_{#1}^{\IfBooleanTF{#4}{\prime #2}{#2}}\ifthenelse{\isempty{#3}}{}{(#3)}}
\DeclareDocumentCommand{\ub}{D<>{} O{}  D(){} t'}{\mathsf{U}_{#1}^{\IfBooleanTF{#4}{\prime #2}{#2}}\ifthenelse{\isempty{#3}}{}{(#3)}}
\DeclareDocumentCommand{\gub}{D<>{} O{}  D(){} t'}{\mathsf{G}_{#1}^{\IfBooleanTF{#4}{\prime #2}{#2}}\ifthenelse{\isempty{#3}}{}{(#3)}}
\DeclareDocumentCommand{\lb}{D<>{} O{}  D(){} t'}{\mathsf{L}_{#1}^{\IfBooleanTF{#4}{\prime #2}{#2}}\ifthenelse{\isempty{#3}}{}{(#3)}}
\DeclareDocumentCommand{\game}{D<>{} O{} D(){} t'}{\mathsf{G}_{#1}^{\IfBooleanTF{#4}{\prime}{}#2}\ifthenelse{\isempty{#3}}{}{(#3)}}
\DeclareDocumentCommand{\transition}{D<>{} O{} D(){}}{\rightarrow_{#1}^{#2}\ifthenelse{\isempty{#3}}{}{(#3)}}


\newcommand{\Ts}{\mathcal{T}}
\newcommand{\MC}{\mathsf{M}}
\newcommand{\Mdp}{\mathcal{M}}
\newcommand{\Pm}{\mathbf{P}}
\newcommand{\SG}{\textrm{SG}}
\newcommand{\SGs}{\textrm{SGs}}
\newcommand{\M}{\mathsf{M}}
\DeclareDocumentCommand{\G}{D<>{} O{} t' D(){}}{\mathsf{G}_{#1}^{\IfBooleanTF{#3}{\prime}{}#2}\ifthenelse{\isempty{#4}}{}{(#4)}}
\DeclareDocumentCommand{\exGame}{D<>{} O{} t'  D(){}}{\mathsf{G}_{#1}^{\IfBooleanTF{#3}{\prime}{}#2}\ifthenelse{\isempty{#4}}{}{(#4)}=(\states<#1>[\IfBooleanTF{#3}{\prime}{}#2],\states<\Box\ifthenelse{\isempty{#1}}{}{,#1}>[\IfBooleanTF{#3}{\prime}{}#2],\states<\circ\ifthenelse{\isempty{#1}}{}{,#1}>[\IfBooleanTF{#3}{\prime}{}#2],\istate<#1>[\IfBooleanTF{#3}{\prime}{}#2],\actions<#1>[\IfBooleanTF{#3}{\prime}{}#2],\Av<#1>[\IfBooleanTF{#3}{\prime}{}#2],\trans<#1>[\IfBooleanTF{#3}{\prime}{}#2])}


\newcommand{\ap}{Ap}
\DeclareDocumentCommand{\states}{D<>{} O{}  t'}{\mathsf{S}_{#1}^{\IfBooleanTF{#3}{\prime~#2}{#2}}}
\DeclareDocumentCommand{\state}{D<>{} O{}  t'}{\mathsf{s}_{#1}^{\IfBooleanTF{#3}{\prime #2}{#2}}}
\DeclareDocumentCommand{\istate}{D<>{} O{} t'}{\mathsf{s}_{0\ifthenelse{\isempty{#1}}{}{,#1}}^{\IfBooleanTF{#3}{\prime~#2}{#2}}}
\newcommand{\inv}{\nu}
\newcommand{\lab}{L}

\newcommand{\edges}{E}
\newcommand{\statesone}{S^N}
\newcommand{\statestwo}{S_2}
\newcommand{\statesp}{S^P}
\newcommand{\initstate}{\state<0>}
\newcommand{\initdist}{\mu}
\DeclareDocumentCommand{\trans}{D<>{} O{} t' D(){} D(){}}{\delta{#1}^{\IfBooleanTF{#3}{\prime}{}#2}\ifthenelse{\isempty{#4}}{}{(#4)}\ifthenelse{\isempty{#5}}{}{(#5)}}
\DeclareDocumentCommand{\Av}{D<>{} O{} t' D(){}}{\mathsf{Av}_{#1}^{\IfBooleanTF{#3}{\prime}{}#2}\ifthenelse{\isempty{#4}}{}{(#4)}}
\newcommand{\rew}{r}
\DeclareDocumentCommand{\F}{D<>{} O{} t' D(){}}{\mathsf{F}_{#1}^{\IfBooleanTF{#3}{\prime}{}#2}\ifthenelse{\isempty{#4}}{}{(#4)}}
\newcommand{\lu}{\hat{l},\hat{u}}

% = paths and strategies
\newcommand{\Path}{\rho}
\DeclareDocumentCommand{\Path}{D<>{} O{} t' D(){}}{\path<#1>[#2]\IfBooleanTF{#3}{'}{}(#4)}
\DeclareDocumentCommand{\path}{D<>{} O{} t' D(){}}{\rho_{#1}^{\IfBooleanTF{#3}{\prime}{}#2}\ifthenelse{\isempty{#4}}{}{(#4)}}
\newcommand{\fpath}{\mathsf{w}}
\DeclareDocumentCommand{\Paths}{D<>{} O{} t' D(){}}{\Omega_{#1}^{\IfBooleanTF{#3}{\prime}{}#2}\ifthenelse{\isempty{#4}}{}{(#4)}}
\newcommand{\straa}{\sigma}
\newcommand{\straas}{\Sigma}
\newcommand{\strab}{\tau}
\newcommand{\strabs}{\Tau}
\newcommand{\plays}{\mathsf{Plays}}
\newcommand{\play}{\mathsf{Play}}
\newcommand{\last}[1]{\text{last}(#1)}%{#1\mathord\downarrow}
\DeclareDocumentCommand{\strategy}{D<>{} O{} D(){}
  t*}{{\IfBooleanTF{#4}{\tau}{\sigma}}_{#1}^{#2}\ifthenelse{\isempty{#3}}{}{(#3)}}

% = probability

\newcommand{\inits}{\hat s}
\DeclareDocumentCommand{\actions}{D<>{} O{} t' d()}{{\IfNoValueTF{#4}{\mathsf{A}}{\fun{\mathsf{A}}{#4}}}_{#1}^{\IfBooleanTF{#3}{\prime~#2}{#2}}}
\DeclareDocumentCommand{\action}{D<>{} O{} t'}{\mathsf{a}_{#1}^{\IfBooleanTF{#3}{\prime#2}{#2}}}
\newcommand{\pat}{\omega}
\newcommand{\Pat}{\mathsf{Runs}}
\newcommand{\fpat}{w}
\newcommand{\mem}{\mathsf{M}}
\newcommand{\Cone}{\mathsf{Cone}}
\newcommand{\calF}{\mathcal{F}}
\newcommand{\mec}{\mathsf{MEC}}
\newcommand{\scc}{\mathsf{SCC}}
\newcommand{\ec} {\mathsf{EC}}
\newcommand{\bscc}{\mathsf{BSCC}}

\newcommand{\attractor}{\mathsf{prob1}}
\newcommand{\pr}{\mathbb P}
\renewcommand{\Pr}[3]{\pr^{#1}\hspace{-0.16em}\left[{#3}\right]}   %\Pr{strat}{state}{event}
\newcommand{\PrS}[3]{\pr^{#1}_{#2}\hspace{-0.16em}\left[{#3}\right]}   %\Pr{strat}{state}{event}
\newcommand{\expected}{\mathbb{E}}
\newcommand{\expsucc}{\expected_\trans}
\newcommand{\Ex}[3]{\expected^{#1}_{#2}\hspace{-0.16em}\left[{#3}\right]}   %\Ex{strat}{state}{f}
\newcommand{\ExS}[3]{\expected^{#1}_{#2}\hspace{-0.16em}\left[{#3}\right]}   %\Ex{strat}{state}{f}


% == LOCAL

\newcommand{\push}{\mathrm{push}}
\newcommand{\pop}{\mathrm{pop}}
\newcommand{\topx}{\mathrm{top}}


\newcommand{\reward}{\vec{r}}
\newcommand{\ex}{\vecl{exp}}
\newcommand{\sat}{\vecl{sat}}
\newcommand{\satscalar}{\mathit{sat}}
\newcommand{\psat}{\vecl{pr}}%{\vec{p^{sat}}}
\newcommand{\psatscalar}{\mathit{pr}}%{\mathit{p^{sat}}}

\newcommand{\lrLim}[1]{\mathrm{lr}(#1)}  %\lr{rew}{run}
\newcommand{\lrSf}[1]{\mathrm{lr}_{\mathrm{sup}}(#1)}  %\lrS{rew}{run}
\newcommand{\lrIf}[1]{\mathrm{lr}_{\mathrm{inf}}(#1)}  %\lrI{rew}{run}
\newcommand{\lrInf}{\mathrm{lr}^{\mathrm{inf}}}  %\lrI{rew}{run}

\newcommand{\appear}{\textsf{Appear}}

\NewDocumentCommand{\spannorm}{m}{\fun{\operatorname{sp}}{#1}}

% == OTHER

\newcommand{\chapterSpace}{\vspace*{1cm}}

% == Tikz
\newcommand{\drawcirc}{\node[draw,circle,minimum size=.7cm, outer sep=1pt]}
\newcommand{\drawbox}{\node[draw,rectangle,minimum size=.7cm, outer sep=1pt]}
\newcommand{\drawdummy}{\node[minimum size=0,inner sep=0]}


\newcommand{\para}[1]{\noindent\textbf{#1} }


\newcommand{\sinks}{\mathsf{Zero}}


% ===== Fruit
% ===== Old but still used
\newcommand{\best}{\mathsf{best}}
\newcommand{\Vis}{\widehat{\states}}
\DeclareDocumentCommand{\target}{D<>{} O{}
  t'}{\mathfrak{1}_{#1}^{\IfBooleanTF{#3}{\prime}{}#2}}
\NewDocumentCommand{\exit}{D<>{}D[]{}D(){}}{\mathsf{bestExit}_{#1}^{#2}\ifthenelse{\isempty{#3}}{}{(#3)}}



% ====== New
\newcommand{\targetset}{\mathsf{T}}
\newcommand{\sink}{\mathfrak{0}} %set of states with no path to target, needed for bellman equations

\newcommand{\Prob}{\mathbb{P}}


\newcommand{\INITIALIZE}{\mathsf{INITIALIZE}}
\newcommand{\UPDATE}{\mathsf{UPDATE}}
\newcommand{\GETSTATES}{\mathsf{GET\_STATES}}
\newcommand{\SIMULATE}{\mathsf{SIMULATE}}
\newcommand{\STUCK}{\mathsf{STUCK}}
\newcommand{\TERMCRIT}{\mathsf{TERM\_CRIT}}
\newcommand{\nextstate}{\mathrm{NEXT\_STATE}}




%%% Local Variables:
%%% mode: latex
%%% TeX-master: "../paper"
%%% End: